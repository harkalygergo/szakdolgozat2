%Az elsõ fejezet
\Chapter{Bevezetés}
\label{Chap:bevezetes}

A napjainkban tapasztalható felgyorsult technikai fejlődés a webes környezetben is azonnali megoldásokat kívánt a programozóktól, részben ezért is született már számos fejlesztőkörnyezet. Az internetnek köszönhetően pedig több, egymástól távol élő szoftverfejlesztő tud együtt dolgozni, valamint az ő munkájukat önkéntes, a programozás iránt érdeklődő személy segíteni, észrevételeikkel, visszajelzéseikkel támogatni. Így születhetett meg számos ingyenes, nagy tudással bíró, interneten keresztül letölthető tartalomkezelő rendszer is, ismertebb nevén CMS, mely az angol Content Management System szavakból álló mozaikszóról kapta a nevét.

Ezek közül a legismertebbek betűsorrendben: Drupal, ExpressionEngine, Joomla, Magento, OpenCart, PrestaShop, WordPress. Emellett természetesen az informatikai vállalkozások többsége is kifejlesztett már saját tartalomkezelő rendszert, melyek kereskedelmi forgalomban nem elérhetőek.

A CMS-ek segítségével akár egy néhány perces, szinte automatikus telepítést követően már kész adminisztrációs felületet, webes szolgáltatásokat, sablonokat és kiegészítőket kap a szoftver használója. Többségük a tartalom felvitelén és módosításán túl olyan további funkciókat is biztosít, mint hozzászólási lehetőség az egyes bejegyzésekhez, médiatár, hírlevélküldés vagy akár kész webáruház kialakítása kosárral és fizetési modulok integrálásával.

A kommunikációs csatornák (telefonhívás, sms, e-mail, honlap, stb.) kiszélesedésével és a webáruházak létrehozásának leegyszerűsödésével lényegében kinyílt a világ a vállalkozások előtt. A széles spektrumú értékesítési csatornák természetesen az ügyfélkör növekedését is magukkal vonták, ezért a korábban működőképes, kizárólag ismeretségi vagy helybeli közelségre alapuló értékesítési megoldások mára kevésbé alkalmazhatóak. A felhasználók, vásárlók naponta rengeteg ingert kapnak, emellett végzettségük, fizikai távolságuk és egyéb jellemzőik alapján teljesen eltérően kezelendőek. Így új megoldások születtek a felhasználók adatainak nyilvántartására és a vállalkozást is támogatni képes kezelésére, melyek napjainkban még kevésbé terjedtek el. Ezeket az eszközöket nevezi a szakma CRM rendszereknek, mely az angol Customer Relationship Management, azaz ügyfélkapcsolat-kezelőre utal. Az egyik leghíresebb magyar megoldás a miniCRM, míg az Oracle hasonló webes megoldása - mely inkább nagyvállalatok körében elterjedt - a Siebel, számos nemzetközi vállalkozás pedig a Salesforce rendszerét használja.

A CRM lényege, hogy a felhasználókról, vásárlókról, illetve a velük történő interakciókról szinte minden adatot és iratot nyilván lehet tartani, azokból statisztikákat készíteni, a statisztikákból pedig célirányosabb marketinget végezni. A személyes adatok mellett akár a szerződések és egyéb dokumentumok feltöltésére, visszakeresésére is alkalmas, továbbá rögzíthetőek a kapcsolatfelvétel időpontjai és eseményei, függetlenül attól, hogy az adott interakciót melyik fél kezdeményezte. Ennek köszönhetően teljes körű ügyféltörténet bontakozik ki a CRM használója előtt, hisz akár pontosan látja például, hogy az adott vevő mikor látogatott el utoljára az üzletbe, ott mi után érdeklődött, milyen kérdései voltak, kötött-e szerződést, ha igen, azon mi szerepel, hívta-e azóta az ügyfélszolgálatot, és még sorolhatnánk.

\Section{Motiváció}

%A meglévő tartalomkezelő rendszerek - bár rendkívül sok opciót foglalnak magukban, mégis - a hosszú távú működésben gondolkodó informatikai vállalkozásoknak nem biztosítanak megfelelő hátteret. Nem véletlen, hogy a vállalatok többsége vagy egyszerre vesz igénybe több apró szolgáltatást, s biztosítja így a működéshez szükséges minden funkciót, vagy saját fejlesztésű megoldásokkal dolgozik, ugyanis a külső, más fejlesztői csoportoktól való függőség számos problémát felvet. Ahhoz, hogy a jelenleg is elérhető CMS-ekkel az ügyfelek igényeit széles körben ki lehessen szolgálni, több kiegészítőt, úgynevezett plugint szükséges telepíteni, vagy saját magunknak fejleszteni. A mások által készített és általunk telepített eszközök azonban nem mindig megfelelően integrálódnak az egyéni érdekek alapján módosított alapverziókhoz.

%Emellett mind a CMS, mind a CRM önmagában olyan bonyolult struktúrájú, hogy alapvetően ezeket külön fejlesztések és szoftverek formájában szokás használni. Bár össze lehet őket kapcsolni, esetleg bővítményként telepíteni egyiket a másikba, azonban nem elterjedt olyan megoldás, ahol a szoftver alapvetően úgy épülne fel, hogy mindkét opciót alapértelmezetten tartalmazza.

%Ezért döntöttem amellett, hogy marketingorientált megközelítést alkalmazva, a meglévő tapasztalatokra alapozva olyan tartalomkezelő rendszert hozok létre, mely egy weboldal és/vagy webáruház tartalmi kiszolgálása mellett képes ellátni az ügyféladatbázis nyilvántartását és kezelését is.

A feladat egy Európai Uniós projekt keretében valósulhatott meg, mely során a "Közép-dunántúli kerékpáros turisztikai hálózat" című pályázatban foglaltaknak megfelelően kellett kialakítani mind a megjelenést, mind a működést. Az elkészült website a \url{http://www.katt.sport.hu} domain név alatt tekinthető meg, mivel a Tatabányai Alpin Sportklub, mint pályázó vezetése a jelenleginél is nagyobb víziókban gondolkodva a Közép-dunántúli Aktív Turisztikai Térség szavakból képzett KATT mozaikszóval illette a projektet.

Az ajánlatkérésben foglalt elvárások meghatározó módon befolyásolták, korlátozták a felhasználható alternatívákat:
\begin{itemize}
	\item A későbbiek folyamán lehessen könnyen továbbfejleszteni az oldalrendszert.
	\item Az oldalrendszer képes legyen kezelni legalább 120.000 karakter szöveget és 2000 db digitális fotót.
	\item A fenntartási időszak végéig vállaljanak garanciát az oldal programozására.
	\item Tudjunk eseményeket megjeleníteni is, és ezeket mi tudjuk frissíteni.
	\item Több admin személyt is kinevezhessünk, és megfelelő jogosultság birtokában frissíthessék az oldal tartalmát.
\end{itemize}

A továbbiakban bemutatandó tartalomkezelők közül a WordPress bizonyult a leginkább megfelelőnek, hiszen már az alaprendszer is képes számos igényt kielégíteni, így például a meglévő tartalmak egyszerű, grafikus felületen keresztüli módosítási lehetőségét, képek feltöltését, módosítását (Médiatár). Mivel a WP a legelterjedtebb nyílt forráskódú CMS, így a későbbi továbbfejlesztés sem jelent majd akadályt, valamint a hosszú távú biztonságos és korszerű fenntartása is biztosított a közösségnek hála. Felhasználói jogosultsági szintekből pedig öt beépítetttel is rendelkezik: feliratkozó (subscriber), közreműködő (contributor), szerző (author), szerkesztő (editor) és adminisztrátor (administrator), szükség esetén azonban további szintek is deklarálhatóak.